\let\negmedspace\undefined
\let\negthickspace\undefined
\documentclass[journal]{IEEEtran}
\usepackage[a5paper, margin=10mm, onecolumn]{geometry}
%\usepackage{lmodern} % Ensure lmodern is loaded for pdflatex
\usepackage{tfrupee} % Include tfrupee package

\setlength{\headheight}{1cm} % Set the height of the header box
\setlength{\headsep}{0mm}     % Set the distance between the header box and the top of the text

\usepackage{gvv-book}
\usepackage{gvv}
\usepackage{cite}
\usepackage{amsmath,amssymb,amsfonts,amsthm}
\usepackage{algorithmic}
\usepackage{graphicx}
\usepackage{textcomp}
\usepackage{xcolor}
\usepackage{txfonts}
\usepackage{listings}
\usepackage{enumitem}
\usepackage{mathtools}
\usepackage{gensymb}
\usepackage{comment}
\usepackage[breaklinks=true]{hyperref}
\usepackage{tkz-euclide} 
\usepackage{listings}
% \usepackage{gvv}                                        
\def\inputGnumericTable{}                                 
\usepackage[latin1]{inputenc}                                
\usepackage{color}                                            
\usepackage{array}                                            
\usepackage{longtable}                                       
\usepackage{calc}                                             
\usepackage{multirow}                                         
\usepackage{hhline}                                           
\usepackage{ifthen}                                           
\usepackage{lscape}
\begin{document}

\bibliographystyle{IEEEtran}
\vspace{3cm}

\title{Definite Integrals and Applications of Integrals}
\author{AI24BTECH11015 - Harshvardhan Patidar}
 \maketitle
% \newpage
% \bigskip
{\let\newpage\relax\maketitle}

\renewcommand{\thefigure}{\theenumi}
\renewcommand{\thetable}{\theenumi}
\setlength{\intextsep}{10pt} % Space between text and floats


\numberwithin{equation}{enumi}
\numberwithin{figure}{enumi}
\renewcommand{\thetable}{\theenumi}



%my_code_starts


Section-B | JEE Main / AIEEE \\ \\
\begin{enumerate}
	\item 
		$\int_0^{10\pi} \abs{\sin \brak{x}} \, dx$ is  
		\begin{enumerate}
			\item 20\item 8\item 10\item 18
		\end{enumerate} 
		\hfill (2002)
	\item
		$I_n = \int \limits_{0}^{\frac{\pi}{4}} \tan^n \brak{x} \, dx$ then $\lim \limits_{n \to \infty} n\brak{I_n + I_{n+2}}$ equals 
				\begin{enumerate}
					\item $\frac{1}{2}$
	 				\item 1
					\item $\infty$
					\item zero
	 			\end{enumerate}
				\hfill(2002)
	\item
		$ \int \limits_0^2 \brak{x^2}dx $ is  
			\begin{enumerate}
				\item $2-\sqrt2$
				 \item $2+\sqrt2$
				\item $\sqrt2 - 1$
				\item $-\sqrt2 - \sqrt3 + 5$
			\end{enumerate}
			\hfill (2002)
	\item 
		$ \int_{-\pi}^{\pi} \frac{2x\brak{1 + \sin \brak{x}}}{1+\cos^2 \brak{x}} \, dx$ is
			\begin{enumerate}
				\itemsep0.3em
					\item $\frac{\pi^2}{4}$
					\item $\pi^2$
					\item zero
					\item $\frac{\pi}{2}$
			\end{enumerate}
			\hfill (2002)
	\item
		If $y=f\brak{x}$ makes +ve intercept of $2$ and $0$ unit on $x$ and $y$ axes and encloses an area of $\frac{3}{4}$ square unit with the axes then $\int \limits_0^2 x f^{\prime}\brak{x}dx$ is
			\begin{enumerate}
				\itemsep0.3em
				\item $\frac{3}{2}$
				\item $1$
				\item $\frac{5}{4}$
				\item $-\frac{3}{4}$
			\end{enumerate}
			\hfill (2002)
	\item
		The area bounded by the curves $y= \ln \brak{x}$ , $y= \ln \brak{\abs{x}}$, $y=\abs{\ln\brak{x}}$ and $y=\abs{\ln \brak{\abs{x}}}$
			\begin{enumerate}
					\item $4 \, sq. \, units$
					\item $6 \, sq. \, units$
					\item $10 \, sq. \, units$
					\item none of these
			\end{enumerate}
			\hfill(2002)
	\item
		The area of the region bounded by the curves $y= \abs{x-1}$ and $y=3-\abs{x}$ is
			\begin {enumerate}
				\item $6 \, sq. \, units$
				\item $2 \, sq. \, units$
				\item $3 \, sq. \, units$
				\item $4 \, sq. \, units$
			\end {enumerate}
			\hfill (2003)
	\item
		If $f\brak{a+b-x}=f\brak{x}$ then $\int \limits_a^b xf\brak{x}dx$ is equal to
			\begin {enumerate}
				\itemsep0.3em
				\item $\frac{a+b}{2} \int \limits_a^b f\brak{a+b+x}dx$
				\item $\frac{a+b}{2} \int \limits_a^b f\brak{b-x}dx$
				\item $\frac{a+b}{2} \int \limits_a^b f\brak{x}dx$
				\item $\frac{b-a}{2} \int \limits_a^b f\brak{x}dx$
			\end {enumerate}
			\hfill (2003)
	\item 
		Let $f\brak{x}$ be a function satisfying $f^{\prime}\brak{x} = f\brak{x}$ with $f\brak{0}=1$ and $g\brak{x}$ be a function that satisfies $f\brak{x} + g\brak{x}=x^2$. Then the value of the integral $\int \limits_0^1 f\brak{x}g\brak{x}dx$, is
			\begin {enumerate}
				\itemsep0.4em
				\item $e+ \frac{e^2}{2} + \frac{5}{2}$
				\item $e- \frac{e^2}{2} - \frac{5}{2}$
				\item $e+ \frac{e^2}{2} - \frac{3}{2}$
				\item $e- \frac{e^2}{2} - \frac{3}{2}$
			\end {enumerate}
			\hfill (2003)
	\item
		The value of the integral $I = \int \limits_0^1 \brak{x}\brak{1-x}^n \, dx$ is
			\begin{enumerate}
				\itemsep0.5em
				\item $\frac{1}{n+1} + \frac{1}{n+2}$
				\item $\frac{1}{n+1}$
				\item $\frac{1}{n+2}$
				\item $\frac{1}{n+1} - \frac{1}{n+2}$
			\end{enumerate}
			\hfill (2003)
	\item
		$\lim \limits_{n \to \infty} \sum \limits_{r=1}^{n} \frac{1}{n} e^{\frac{r}{n}}$ is
			\begin{enumerate}
				\item $e+1$
				\item $e-1$
				\item $1-e$
				\item $e$	
			\end{enumerate}
			\hfill (2004)
	\item 
		The value of $\int \limits_{-2}^{3} \abs{1-x^2}dx$ is 
			\begin{enumerate}
				\itemsep0.4em
				\item $\frac{1}{3}$
				\item $\frac{14}{3}$
				\item $\frac{7}{3}$
				\item $\frac{28}{3}$
			\end{enumerate}
			\hfill (2004)
	\item
		The value of $I = \int \limits_0^{\frac{\pi}{2}} \frac{(\sin \brak{x} + \cos \brak{x})^2}{\sqrt{1+ \sin \brak{2x}}} dx$ is
			\begin{enumerate}
				\item $3$
				\item $1$
				\item $2$
				\item $0$
			\end{enumerate}
			\hfill (2004)
	\item
		If $\int \limits_{0}^{\pi} x f\brak{\sin \brak{x}}dx = A \int \limits_{0}^{\frac{\pi}{2}} f\brak{\sin \brak{x}}dx$, then $A$ is 
			\begin{enumerate}
				\itemsep0.3em
				\item $2\pi$
				\item $\pi$
				\item $\frac{\pi}{4}$
				\item $0$
			\end{enumerate}
			\hfill (2004)
	\item
		If $f\brak{x}= \frac{e^x}{1+e^x}, \, I_1=\int \limits_{f\brak{-a}}^{f\brak{a}}\brak{x} g\brak{x\brak{1-x}}dx$ and $I_2 = \int \limits_{f\brak{-a}}^{f\brak{a}}g\brak{x\brak{1-x}}dx$, then the value of $\frac{I_2}{I_1}$ is 
			\begin{enumerate}
				\item $1$
				\item $-3$
				\item $-1$
				\item $2$
			\end{enumerate}
			\hfill (2004)
\end{enumerate}



  
\end{document}


