%iffalse
\let\negmedspace\undefined
\let\negthickspace\undefined
\documentclass[journal,12pt,twocolumn]{IEEEtran}
\usepackage{cite}
\usepackage{amsmath,amssymb,amsfonts,amsthm}
\usepackage{algorithmic}
\usepackage{graphicx}
\usepackage{textcomp}
\usepackage{xcolor}
\usepackage{txfonts}
\usepackage{listings}
\usepackage{enumitem}
\usepackage{mathtools}
\usepackage{gensymb}
\usepackage{comment}
\usepackage[breaklinks=true]{hyperref}
\usepackage{tkz-euclide} 
\usepackage{listings}
\usepackage{gvv}                                        
%\def\inputGnumericTable{}                                 
\usepackage[latin1]{inputenc}                                
\usepackage{color}                                            
\usepackage{array}                                            
\usepackage{longtable}                                       
\usepackage{calc}                                             
\usepackage{multirow}                                         
\usepackage{hhline}                                           
\usepackage{ifthen}                                           
\usepackage{lscape}
\usepackage{tabularx}
\usepackage{array}
\usepackage{float}


\newtheorem{theorem}{Theorem}[section]
\newtheorem{problem}{Problem}
\newtheorem{proposition}{Proposition}[section]
\newtheorem{lemma}{Lemma}[section]
\newtheorem{corollary}[theorem]{Corollary}
\newtheorem{example}{Example}[section]
\newtheorem{definition}[problem]{Definition}
\newcommand{\BEQA}{\begin{eqnarray}}
\newcommand{\EEQA}{\end{eqnarray}}
\newcommand{\define}{\stackrel{\triangle}{=}}
\theoremstyle{remark}
\newtheorem{rem}{Remark}

% Marks the beginning of the document
\begin{document}
\bibliographystyle{IEEEtran}
\vspace{3cm}

\title{Matrices and Determinants}
\author{AI24BTECH11015 - Harshvardhan Patidar}
\maketitle
\newpage
\bigskip

\renewcommand{\thefigure}{\theenumi}
\renewcommand{\thetable}{\theenumi}


%code starts

Section-A | JEE Advanced/ IIT-JEE
 

		\section{Fill in the Blanks:}
			\begin{enumerate}
				\item
					How many $3 \times 3$ matrices $M$ with entries from $\brak{0,1,2}$ are there, for which the sum of the diagonal entries of $M^TM$ is $5$?
						\begin{enumerate}
							\item $126$
							\item $198$
							\item $162$
							\item $135$
						\end{enumerate}
						\hfill (JEE Adv. 2017)
					\item
					Let $M= \mydet{
						\sin^4 \brak{\theta} & -1 -\sin^2 \brak{\theta}\\
						1+\cos^2 \brak{\theta} & \cos^4 \brak{\theta}
						} = 
						\alpha I + \beta M^{-1}$\\
					Where $\alpha = \alpha \brak{\theta}$ and $\beta = \beta \brak{\theta}$ are real numbers, and $I$ is the $2 \times 2$ identity matrix. If $\mathrm{a^*}$ is the minimum of the set $\brak{\alpha \brak{\theta} : \theta \in [0, 2 \pi)}$ and $\mathrm{b^*}$ is the minimum of the set $\brak{\beta \brak{\theta} : \theta \in [0, 2 \pi)}$. Then the value of $\mathrm{a^*} + \mathrm{b^*}$ is
						\begin{enumerate}
							\item $-\frac{31}{16}$
							\item $-\frac{17}{16}$
							\item $-\frac{37}{16}$
							\item $-\frac{29}{16}$
						\end{enumerate}
						\hfill (JEE Adv. 2019)
			\end{enumerate}

		\section{MCQs with More than One Correct}
			\begin{enumerate}
				\item
					The determinant $
							\mydet{
								a & b & a \alpha + b\\
								b & c & b \alpha + c\\
								a \alpha + b & b \alpha + c & 0
							}
					$ is equal to zero, if
						\begin{enumerate}
									\item $a,b,c$ are in A.P.
									\item $a,b,c$ are in G.P.
									\item $a,b,c$ are in H.P.
									\item $\alpha$ is a root of the equation $ax^2 + bx +c=0$
									\item $\brak{x-\alpha}$ is a factor of $ax^2 + bx +c$
						\end{enumerate}
						\hfill (1986-2 Marks)
				\item 
					If $
					\mydet{
						6i & -3i & 1\\
						4 & 3i & -1\\
						20 & 3 & i
					} = x +iy$, then 
					\begin{enumerate}
						\item $x=3,y=1$
						\item $x=1,y=3$
						\item $x=0,y=3$
						\item $x=0,y=0$
					\end{enumerate}
					\hfill (1998-2 Marks)
				\item 
					Let $M$ and $N$ be two $3 \times 3$ non-singulr skew-symmetric matrices such that $MN=NM$. If $P^T$ denotes the transpose of $P$, then $M^2N^2\brak{M^TN^{-1}}^{-1} \brak{MN^{-1}}^T$ is equal to 
					\begin{enumerate}
						\item $M^2$
						\item $-N^2$
						\item $-M^2$
						\item $MN$
					\end{enumerate}
					\hfill (2011)
				\item 
					If the adjoint of a $3 \times 3$ matrix $P$ is 
							$\myvec{	
								1&4&4\\
								2&1&7\\
								1&1&3
							}$
					, then the possible value(s) of the determinant of $P$ is (are)
						\begin{enumerate}
							\item $-2$
							\item $-1$
							\item $1$
							\item $2$
						\end{enumerate}
						\hfill (2012)
				\item 
					For $3 \times 3$ matrices $M$ and $N$, which of the following statement(s) is (are) NOT correct?
						\begin{enumerate}
							\item $N^TMN$ is symmetric or skew symmetric, according as $M$ is symmetric or skew symmetric
							\item $MN-NM$ is skew symmetric for all matrices $M$ and $N$.
							\item $MN$ is symmetric for all symmetric matrices $M$ and $N$.
							\item (adj$M$)(adj$N$) = adj($MN$) for all invertible matrices $M$ and $N$.
						\end{enumerate}
						\hfill (JEE Adv. 2013)
				\item 
					Let $\omega$ be a complex cube root of unity with $\omega \neq 1 $ and $P={p_{ij}}$ be a $n \times n$ matrix with $p_{ij} = \omega^{i+j}$. Then $p^2 \neq 0$, when $n=$
					\begin{enumerate}
						\item $57$
						\item $55$
						\item $58$
						\item $56$
					\end{enumerate}
					\hfill (JEE Adv. 2013)
				\item 
					Let $M$ be a $2 \times 2$ symmetric matrix with integer entries. Then $M$ is invertible if
						\begin{enumerate}
							\item The first column of $M$ is the transpose of the second row of $M$
							\item The second row of $M$ is the transpose of the first column of $M$
							\item $M$ is a diagonal matrix with non-zero entries in the main diagonal
							\item The product of entries in the main diagonal of $M$ is not the square of an integer
						\end{enumerate}
						\hfill (JEE Adv. 2014)
				\item
					Let $M$ and $N$ be two $3 \times 3$ matrices such that $MN=NM$. Further, if $M \neq N^2$ and $M^2 = N^4$, then
						\begin{enumerate}
							\item determinant of $(M^2 + N^2)$ is $0$
							\item there is $3 \times 3$ non-zero matrix $U$ such that $(M^2+MN^2)U$ is the zero matrix
							\item determinant of $(M^2 + MN^2) \geq 1$
							\item determinant of $(M^2 + MN^2)U$ equals the zero matrix then $U$ is the zero matrix
						\end{enumerate}
						\hfill (JEE Adv. 2014)
				\item
					Which of the following values of $\alpha$ satisfy the equation
						$\mydet{
							(1 + \alpha)^2 & (1 + 2\alpha)^2 & (1+3\alpha)^2\\
							(2 + \alpha)^2 & (2 + 2\alpha)^2 & (2+3\alpha)^2\\
							(3 + \alpha)^2 & (3 + 2\alpha)^2 & (3+3\alpha)^2
						} = -648 \alpha $ ?
							\begin{enumerate}
								\item $-4$
								\item $9$
								\item $-9$
								\item $4$
							\end{enumerate}
							\hfill (JEE Adv. 2015)
				\item
					Let $X$ and $Y$ be two arbitrary, $3 \times 3$, non-zero, skew-symmetric matrices and $Z$ be an arbitrary $3 \times 3$, non-zero, symmetric matrix. Then which of the following matrices is (are) skew symmetric?
					\begin{enumerate}
						\item $Y^3Z^4 -Z^4Y^3$
						\item $X^{44} + Y^{44}$
						\item $X^4Z^3 -Z^3X^4$
						\item $X^{23} + Y^{23}$
					\end{enumerate}
					\hfill (JEE Adv. 2015)
				\item 
					Let $P = 
						\myvec{		
							3 & -1 & -2\\
							2 & 0 & \alpha\\
							3 & -5 & 0
						}$,
					where $\alpha \in \mathbb{R}$. Suppose $Q= \mathrm{\sbrak{q_{ij}}}$ is a matrix such that $PQ=kI$, where $k \in \mathbb{R}, k \neq 0$ and $I$ is the identity matrix of order $3$. If $q_{23} = -\frac{k}{8}$ and det\brak{Q}$= \frac{k^2}{2}$, then
					\begin{enumerate}
						\item $a=0, k=8$
						\item $4a-k+8=0$
						\item det $\brak{P adj \brak{Q}} = 2^9$
						\item det $\brak{Q adj \brak{P}} = 2^{13}$
					\end{enumerate}
					\hfill (JEE Adv. 2016)
				\item
					Let $a, \lambda, \mu, \in \mathbb{R}$. Consider the system of linear equations $$ax+2y=\lambda$$ $$3x-2y=\mu$$ Which of the following statement(s) is (are) correct?
					\begin{enumerate}
						\item If $a=-3$, then the system has infinitely many solutions for all value of $\lambda$ and $\mu$.
						\item If $a \neq -3$, then the system has unique solution for all values of $\lambda$ and $\mu$.
						\item If $\lambda + \mu = 0$, then the system has infinitely many solutions for $a = -3$.
						\item If $\lambda + \mu \neq 0$, then the system has no solution for $a = -3$
					\end{enumerate}
					\hfill (JEE Adv. 2016)
				\item 
					Which of the following is (are) not the square of a $3 \times 3$ matrix with real entries?
						\begin{enumerate}
							\item 
								$\myvec{
									1&0&0\\
									0&1&0\\
									0&0&1
								}$\\
							\item 
								$\myvec{
									1&0&0\\
									0&1&0\\
									0&0&-1
								}$\\
							\item 
								$\myvec{
									1&0&0\\
									0&-1&0\\
									0&0&-1
								}$\\
							\item 
								$\myvec{
									-1&0&0\\
									0&-1&0\\
									0&0&-1
								}$\\
						\end{enumerate}
						\hfill (JEE Adv. 2017)	
			\end{enumerate}
\end{document}
