%iffalse
\let\negmedspace\undefined
\let\negthickspace\undefined
\documentclass[journal,12pt,twocolumn]{IEEEtran}
\usepackage{cite}
\usepackage{amsmath,amssymb,amsfonts,amsthm}
\usepackage{algorithmic}
\usepackage{graphicx}
\usepackage{textcomp}
\usepackage{xcolor}
\usepackage{txfonts}
\usepackage{listings}
\usepackage{enumitem}
\usepackage{mathtools}
\usepackage{gensymb}
\usepackage{comment}
\usepackage[breaklinks=true]{hyperref}
\usepackage{tkz-euclide} 
\usepackage{listings}
\usepackage{gvv}                                        
%\def\inputGnumericTable{}                                 
\usepackage[latin1]{inputenc}                                
\usepackage{color}                                            
\usepackage{array}                                            
\usepackage{longtable}                                       
\usepackage{calc}                                             
\usepackage{multirow}                                         
\usepackage{hhline}                                           
\usepackage{ifthen}                                           
\usepackage{lscape}
\usepackage{tabularx}
\usepackage{array}
\usepackage{float}


\newtheorem{theorem}{Theorem}[section]
\newtheorem{problem}{Problem}
\newtheorem{proposition}{Proposition}[section]
\newtheorem{lemma}{Lemma}[section]
\newtheorem{corollary}[theorem]{Corollary}
\newtheorem{example}{Example}[section]
\newtheorem{definition}[problem]{Definition}
\newcommand{\BEQA}{\begin{eqnarray}}
\newcommand{\EEQA}{\end{eqnarray}}
\newcommand{\define}{\stackrel{\triangle}{=}}
\theoremstyle{remark}
\newtheorem{rem}{Remark}

% Marks the beginning of the document
\begin{document}
\bibliographystyle{IEEEtran}
\vspace{3cm}

\title{Matrices and Determinants}
\author{AI24BTECH11015 - Harshvardhan Patidar}
\maketitle
\newpage
\bigskip

\renewcommand{\thefigure}{\theenumi}
\renewcommand{\thetable}{\theenumi}


%code starts

Section-A | JEE Advanced/ IIT-JEE
 
\begin{enumerate} \setcounter{enumi}{2}
	\item
		Fill in the Blanks:
			\begin{enumerate} \setcounter{enumii}{21}
				\item
					How many $3 \times 3$ matrices M with entries from $\brak{0,1,2}$ are there, for which the sum of the diagonal entries of $\mathrm{M^TM}$ is $5$?
						\begin{enumerate}
							\item 126
							\item 198
							\item 162
							\item 135
						\end{enumerate}
						\hfill \brak{JEE Adv. 2017}\\
				\item
					Let $M= \begin{bmatrix}
						\sin^4 \theta & -1 -\sin^2 \theta\\
						1+\cos^2 \theta & \cos^4 \theta
						\end{bmatrix} = 
						\alpha I + \beta M^{-1}$\\
					Where $\alpha = \alpha \brak{\theta}$ and $\beta = \beta \brak{\theta}$ are real numbers, and I is the $2 \times 2$ identity matrix. If $\mathrm{a^*}$ is the minimum of the set $\brak{\alpha \brak{\theta} : \theta \in [0, 2 \pi)}$ and $\mathrm{b^*}$ is the minimum of the set $\brak{\beta \brak{\theta} : \theta \in [0, 2 \pi)}$. Then the value of $\mathrm{a^*} + \mathrm{b^*}$ is
						\begin{enumerate}
							\item $-\frac{31}{16}$
							\item $-\frac{17}{16}$
							\item $-\frac{37}{16}$
							\item $-\frac{29}{16}$
						\end{enumerate}
						\hfill \brak{JEE Adv. 2019}\\
			\end{enumerate}

	\item 
		MCQs with More than One Correct
			\begin{enumerate}
				\item
					The determinant $
							\begin{vmatrix}
								a & b & a \alpha + b\\
								b & c & b \alpha + c\\
								a \alpha + b & b \alpha + c & 0
							\end{vmatrix}
					$ is equal to zero, if
						\begin{enumerate}
									\item $a,b,c$ are in A.P.
									\item $a,b,c$ are in G.P.
									\item $a,b,c$ are in H.P.
									\item $\alpha$ is a root of the equation $ax^2 + bx +c=0$
									\item $\brak{x-\alpha}$ is a factor of $ax^2 + bx +c$
						\end{enumerate}
						\hfill \brak{1986-2 Marks}
				\item 
					If $
					\begin{vmatrix}
						6i & -3i & 1\\
						4 & 3i & -1\\
						20 & 3 & i
					\end{vmatrix} = x +iy$, then 
					\begin{enumerate}
						\item $x=3,y=1$
						\item $x=1,y=3$
						\item $x=0,y=3$
						\item $x=0,y=0$
					\end{enumerate}
					\hfill \brak{1998-2 Marks}\\
				\item 
					Let M and N be two $3 \times 3$ non-singulr skew-symmetric matrices such that $\mathrm{MN=NM}$. If $\mathrm{P^T}$ denotes the transpose of P, then $\mathrm{M^2N^2\brak{M^TN^{-1}}^{-1} \brak{MN^{-1}}^T}$ is equal to 
					\begin{enumerate}
						\item $\mathrm{M^2}$
						\item $-\mathrm{N^2}$
						\item $-\mathrm{M^2}$
						\item MN
					\end{enumerate}
					\hfill \brak{2011}\\
				\item 
					If the adjoint of a $3 \times 3$ matrix $P$ is 
							$\begin{bmatrix}
								1&4&4\\
								2&1&7\\
								1&1&3
							\end{bmatrix}$
					, then the possible value\brak{s} of the determinant of $P$ is \brak{are}
						\begin{enumerate}
							\item $-2$
							\item $-1$
							\item $1$
							\item $2$
						\end{enumerate}
						\hfill \brak{2012}\\
				\item 
					For $3 \times 3$ matrices M and N, which of the following statement\brak{s} is \brak{are} NOT correct?
						\begin{enumerate}
							\item $\mathrm{N^TMN}$ is symmetric or skew symmetric, according as M is symmetric or skew symmetric
							\item MN-NM is skew symmetric for all matrices M and N.
							\item MN is symmetric for all symmetric matrices M and N.
							\item \brak{adjM}\brak{adjN} = adj\brak{MN} for all invertible matrices M and N.
						\end{enumerate}
						\hfill \brak{JEE Adv. 2013}\\
				\item 
					Let $\omega$ be a complex cube root of unity with $\omega \neq 1 $ and $P={p_{ij}}$ be a $n \times n$ matrix with $p_{ij} = \omega^{i+j}$. Then $p^2 \neq 0$,  
			\end{enumerate}
\end{enumerate}


\end{document}
