\let\negmedspace\undefined
\let\negthickspace\undefined
\documentclass[journal]{IEEEtran}
\usepackage[a5paper, margin=10mm, onecolumn]{geometry}
%\usepackage{lmodern} % Ensure lmodern is loaded for pdflatex
\usepackage{tfrupee} % Include tfrupee package

\setlength{\headheight}{1cm} % Set the height of the header box
\setlength{\headsep}{0mm}     % Set the distance between the header box and the top of the text

\usepackage{gvv-book}
\usepackage{gvv}
\usepackage{cite}
\usepackage{amsmath,amssymb,amsfonts,amsthm}
\usepackage{algorithmic}
\usepackage{graphicx}
\usepackage{textcomp}
\usepackage{xcolor}
\usepackage{txfonts}
\usepackage{listings}
\usepackage{enumitem}
\usepackage{mathtools}
\usepackage{gensymb}
\usepackage{comment}
\usepackage[breaklinks=true]{hyperref}
\usepackage{tkz-euclide} 
\usepackage{listings}
% \usepackage{gvv}                                        
\def\inputGnumericTable{}                                 
\usepackage[latin1]{inputenc}                                
\usepackage{color}                                            
\usepackage{array}                                            
\usepackage{longtable}                                       
\usepackage{calc}                                             
\usepackage{multirow}                                         
\usepackage{hhline}                                           
\usepackage{ifthen}                                           
\usepackage{lscape}
\begin{document}

\bibliographystyle{IEEEtran}
\vspace{3cm}

\title{5th September 2020 Shift 1}
\author{AI24BTECH11015 - Harshvardhan Patidar}
 \maketitle
% \newpage
% \bigskip
{\let\newpage\relax\maketitle}

\renewcommand{\thefigure}{\theenumi}
\renewcommand{\thetable}{\theenumi}
\setlength{\intextsep}{10pt} % Space between text and floats


\numberwithin{equation}{enumi}
\numberwithin{figure}{enumi}
\renewcommand{\thetable}{\theenumi}

\begin{enumerate}
    \item If $3 ^ {2 \sin 2\alpha - 1}$, $14$, and $3 ^ {4 - 2 \sin 2\alpha}$ are the first three terms of an A.P. for some $\alpha$, then the sixth term of this A.P. is:
    \begin{enumerate}
        \item $66$
        \item $65$
        \item $81$
        \item $78$
    \end{enumerate}

    \item If the function 
    \begin{equation*}
    f\brak{x} = 
    \begin{cases}
    k_1 \brak{x-\pi}^2 -1, & x \leq \pi \\
    k_2 \cos x, & x > \pi
    \end{cases}
    \end{equation*}
    is twice differentiable, then the ordered pair $\brak{k_1, k_2}$ is equal to:
    \begin{enumerate}
        \item $\brak{\frac{1}{2}, 1}$
        \item $\brak{1, 1}$
        \item $\brak{\frac{1}{2}, -1}$
        \item $\brak{1, 0}$
    \end{enumerate}

    \item If the common tangent to the parabolas $y^2 = 4x$ and $x^2 = 4y$ also touches the circle $x^2 + y^2 = c^2$, then $c$ is equal to:
    \begin{enumerate}
        \item $\frac{1}{2}$
        \item $\frac{1}{2\sqrt{2}}$
        \item $\frac{1}{\sqrt{2}}$
        \item $\frac{1}{4}$
    \end{enumerate}

    \item The negation of the Boolean expression $x \leftrightarrow \sim y$ is equivalent to:
    \begin{enumerate}
        \item $\brak{\sim x \land y} \lor \brak{\sim x \land \sim y}$
        \item $\brak{x \land \sim y} \lor \brak{\sim x \land y}$
        \item $\brak{x \land y} \lor \brak{\sim x \land \sim y}$
        \item $\brak{x \land y} \land \brak{\sim x \lor \sim y}$
    \end{enumerate}

    \item If the volume of a parallelepiped whose coterminus edges are given by vectors $\vec{a} = \hat{i} + \hat{j} + n \hat{k}$, $\vec{b} = 2 \hat{i} + 4 \hat{j} + n \hat{k}$, $\vec{c} = \hat{i} + n \hat{j} + 3 \hat{k}, \brak{n \geq 0}$ is $158$ cu. units, then:
    \begin{enumerate}
        \item $\vec{a} \cdot \vec{c} = 17$
        \item $\vec{b} \cdot \vec{c} = 10$
        \item $n = 7$
        \item $n = 9$
    \end{enumerate}

    \item If $y = y\brak{x}$ is the solution of the differential equation $\frac{5 + e^x}{2+y} \cdot \frac{dy}{dx} + e^x = 0$ satisfying $y\brak{0} = 1$, then a value of $y\brak{\log_e 13}$ is:
    \begin{enumerate}
        \item $1$
        \item $-1$
        \item $2$
        \item $0$
    \end{enumerate}

    \item A survey shows that $73\%$ of the people working in an office like coffee, whereas $65\%$ like tea. If $x$ denotes the percentage of people who like both coffee and tea, then $x$ cannot be:
    \begin{enumerate}
        \item $63$
        \item $38$
        \item $54$
        \item $36$
    \end{enumerate}

    \item The product of the roots of the equation $9x^2 - 18|x| + 5 = 0$ is:
    \begin{enumerate}
        \item $\frac{25}{9}$
        \item $\frac{25}{81}$
        \item $\frac{5}{27}$
        \item $\frac{5}{9}$
    \end{enumerate}

    \item If $\int \brak{e^{2x} + 2e^x - e^{-x} - 1} e^{\brak{e^x + e^{-x}}} dx = g\brak{x} e^{\brak{e^x + e^{-x}}} + c $, where $c$ is the constant of integration, then $g\brak{0}$ is equal to:
    \begin{enumerate}
        \item $2$
        \item $e^2$
        \item $e$
        \item $1$
    \end{enumerate}

    \item If the minimum and the maximum values of the function $f : \sbrak{\frac{\pi}{4}, \frac{\pi}{2}} \to \mathbb{R}$, defined by:

    $$
    f\brak{\theta} = \mydet{
    -\sin^2 \theta & -1 - \sin^2 \theta & 1 \\
    -\cos^2 \theta & -1 - \cos^2 \theta & 1 \\
    12 & 10 & -2
    }
    $$
    
    are $m$ and $M$ respectively, then the ordered pair $\brak{m, M}$ is equal to:
    
    \begin{enumerate}
        \item $\brak{0, 4}$
        \item $\brak{-4, 4}$
        \item $\brak{0, 2\sqrt{2}}$
        \item $\brak{-4, 0}$
    \end{enumerate}
    

    \item Let $\lambda \in \mathbb{R}$. The system of linear equations:
    \begin{align*}
    2x_1 - 4x_2 + \lambda x_3 = 1\\
    x_1 - 6x_2 + x_3 = 2\\
    \lambda x_1 - 10x_2 + 4x_3 = 3\\
    \end{align*}
    is inconsistent for:
    \begin{enumerate}
        \item exactly one negative value of $\lambda$
        \item exactly one positive value of $\lambda$
        \item every value of $\lambda$
        \item exactly two values of $\lambda$
    \end{enumerate}

    \item If $S$ is the sum of the first $10$ terms of the series $\tan^{-1} \frac{1}{3} + \tan^{-1} \frac{1}{7} + \tan^{-1} \frac{1}{13} + \tan^{-1} \frac{1}{21} + \dots$ then $\tan\brak{S}$ is equal to:
    \begin{enumerate}
        \item $\frac{5}{11}$
        \item $-\frac{6}{5}$
        \item $\frac{10}{11}$
        \item $\frac{5}{6}$
    \end{enumerate}

    \item If four complex numbers $z$, $\bar{z},\bar{z} - 2 \text{Re}\brak{\bar{z}}$ and $z - 2 \text{Re}\brak{z}$ represent the vertices of a square of side $4$ units in the Argand plane, then $\abs{z}$ is equal to:
    \begin{enumerate}
        \item $4$
        \item $2$
        \item $4\sqrt{2}$
        \item $2\sqrt{2}$
    \end{enumerate}

    \item If the point $P$ on the curve $4x^2 + 5y^2 = 20$ is farthest from the point $Q\brak{0, -4}$, then $PQ^2$ is equal to:
    \begin{enumerate}
        \item $21$
        \item $36$
        \item $48$
        \item $29$
    \end{enumerate}

    \item The mean and variance of $7$ observations are $8$ and $16$, respectively. If five observations are $2$, $4$, $10$, $12$, and $14$, then the absolute difference of the remaining two observations is:
    \begin{enumerate}
        \item $2$
        \item $4$
        \item $3$
        \item $1$
    \end{enumerate}
\end{enumerate}



  
\end{document}


