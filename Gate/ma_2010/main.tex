\let\negmedspace\undefined
\let\negthickspace\undefined
\documentclass[journal]{IEEEtran}
\usepackage[a5paper, margin=10mm, onecolumn]{geometry}
%\usepackage{lmodern} % Ensure lmodern is loaded for pdflatex
\usepackage{tfrupee} % Include tfrupee package

\setlength{\headheight}{1cm} % Set the height of the header box
\setlength{\headsep}{0mm}     % Set the distance between the header box and the top of the text

\usepackage{gvv-book}
\usepackage{gvv}
\usepackage{cite}
\usepackage{amsmath,amssymb,amsfonts,amsthm}
\usepackage{algorithmic}
\usepackage{graphicx}
\usepackage{textcomp}
\usepackage{xcolor}
\usepackage{txfonts}
\usepackage{listings}
\usepackage{enumitem}
\usepackage{mathtools}
\usepackage{gensymb}
\usepackage{comment}
\usepackage[breaklinks=true]{hyperref}
\usepackage{tkz-euclide} 
\usepackage{listings}
% \usepackage{gvv}                                        
\def\inputGnumericTable{}                                 
\usepackage[latin1]{inputenc}                                
\usepackage{color}                                            
\usepackage{array}                                            
\usepackage{longtable}                                       
\usepackage{calc}                                             
\usepackage{multirow}                                         
\usepackage{hhline}                                           
\usepackage{ifthen}                                           
\usepackage{lscape}
\begin{document}

\bibliographystyle{IEEEtran}
\vspace{3cm}

\title{MA - 2010}
\author{AI24BTECH11015 - Harshvardhan Patidar}
 \maketitle
% \newpage
% \bigskip
{\let\newpage\relax\maketitle}

\renewcommand{\thefigure}{\theenumi}
\renewcommand{\thetable}{\theenumi}
\setlength{\intextsep}{10pt} % Space between text and floats


\numberwithin{equation}{enumi}
\numberwithin{figure}{enumi}
\renewcommand{\thetable}{\theenumi}

\section*{Linked Answer Questions}
\subsection*{Statement for Linked Answer Questions 52 and \ref{53}:}
    For a differentiable function $f\brak{x}$, the integral $\int_{a}^{b} f\brak{x}dx $ is approximated by the formula $h\sbrak{ a_0 f\brak{0} + a_1 f\brak{h}} + h^2 \sbrak{b_0 f^{\prime}\brak{0} + b_1 f^{\prime}\brak{h}}$, which is exact for all polynomials of degree at most $3$.
\begin{enumerate}


    \item \label{53}  The values of $a_0$ and $b_0$ respectively are 
        \begin{enumerate}
            \item $\frac{1}{2}$ and $\frac{1}{2}$
            \item $\frac{1}{12}$ and $-\frac{1}{12}$
            \item $\frac{1}{2}$ and $\frac{1}{12}$
            \item $\frac{1}{2}$ and $-\frac{1}{12}$
        \end{enumerate}

    \subsection*{Statement for Linked Answer Questions \ref{54} and \ref{55}:}
    Let $X = C\sbrak{0,1}$ with the inner product $\langle x,y \rangle = \int_{0}^{1} x\brak{t} y\brak{t} dt,$ $x,y \in C\sbrak{0,1}. X_0 = \cbrak{ x \in X : 
    \int_0^1 t^2 x \brak{t} dt = 0},$ and $X_0^{\perp}$ be the orthogonal complement of $X_0$.

    \item \label{54} Which one of the following statements is correct?
        \begin{enumerate}
            \item Both $X_0$ and $X_0^{\perp}$ are complete
            \item Neither $X_0$ nor $X_0^{\perp}$ is complete
            \item $X_0$ is complete but $X_0^{\perp}$ is not complete
            \item $X_0^{\perp}$ is complete but $X_0$ is not complete
        \end{enumerate}

    \item \label{55} Let $y\brak{t} = t^3, t \in \sbrak{0,1}$ and $x_0 \in X_0^{\perp}$ be the best approximation of $y$. Then $x_0\brak{t}, t \in \sbrak{0,1},$ is
        \begin{enumerate}
            \item $\frac{4}{5} t^2$
            \item $\frac{5}{6} t^2$
            \item $\frac{6}{7} t^2$
            \item $\frac{7}{8} t^2$
        \end{enumerate}

        \section*{General Aptitude (GA) Questions}
        \subsection*{Q.\ref{56} - Q.\ref{60} carry one mark each.}

        \item\label{56} \textit{Which of the following options is the closest in meaning to the word below:} \\ \textbf{Circuitous}
            \begin{enumerate}
                \item cyclic
                \item indirect
                \item confusing
                \item crooked
            \end{enumerate}

            \item \textit{The question below consists of a pair of related words followed by four pairs of words. Select the pair that best expresses the relation in the original pair.} \\
            \textbf{Unemployed : Worker}
            \begin{enumerate}
                \item fallow : land
                \item unaware : sleeper
                \item wit : jester
                \item renovated : house
            \end{enumerate}
    
            \item \textit{Choose the most appropriate word from the options given below to complete the following sentence:} \\
            \textbf{If we manage to \_\_\_\_\_\_\_\_ our natural resources, we would leave a better planet for our children.}
            \begin{enumerate}
                \item uphold
                \item restrain
                \item cherish
                \item conserve
            \end{enumerate}

            \item \textit{Choose the most appropriate word from the options given below to complete the following sentence:} \\
            \textbf{His rather casual remarks on politics \_\_\_\_\_\_\_\_ his lack of seriousness about the subject.}
            \begin{enumerate}
                \item masked
                \item belied
                \item betrayed
                \item suppressed
            \end{enumerate}

            \item \label{60}$25$ persons are in a room. $15$ of them play hockey. $17$ of them play football and $10$ of them play both hockey and football. Then the number of persons playing neither hockey nor football is:
            \begin{enumerate}
                \item $2$
                \item $17$
                \item $13$
                \item $3$
            \end{enumerate}

    \subsection*{Q.\ref{61} - Q.\ref{65} carry two marks each.}

    \item \label{61}\textbf{Modern warfare has changed from large scale clashes of armies to suppression of civilian populations. Chemical agents that do their work silently appear to be suited to such warfare; and regretfully, there exist people in military establishments who think that chemical agents are useful tools for their cause.} \\ \\
    \textit{Which of the following statements best sums up the meaning of the above passage:}
    \begin{enumerate}
        \item Modern warfare has resulted in civil strife.
        \item Chemical agents are useful in modern warfare.
        \item Use of chemical agents in warfare would be undesirable.
        \item People in military establishments like to use chemical agents in war.
    \end{enumerate}

    \item If $137 + 276 = 435$ how much is $731 + 672$?
    \begin{enumerate}
        \item $534$
        \item $1403$
        \item $1623$
        \item $1513$
    \end{enumerate}

    \item $5$ skilled workers can build a wall in $20$ days; $8$ semi-skilled workers can build a wall in $25$ days; $10$ unskilled workers can build a wall in $30$ days. If a team has $2$ skilled, $6$ semi-skilled and $5$ unskilled workers, how long will it take to build the wall?
    \begin{enumerate}
        \item $20$ days
        \item $18$ days
        \item $16$ days
        \item $15$ days
    \end{enumerate}

    \item Given digits $2, 2, 3, 3, 3, 4, 4, 4, 4$ how many distinct $4$ digit numbers greater than $3000$ can be formed?
    \begin{enumerate}
        \item $50$
        \item $51$
        \item $52$
        \item $54$
    \end{enumerate}

    \item \label{65} Hari (H), Gita (G), Irfan (I) and Saira (S) are siblings (i.e. brothers and sisters). All were born on $1^{\text{st}}$ January. The age difference between any two successive siblings (that is born one after another) is less than $3$ years. Given the following facts:
    \begin{enumerate}
        \item Hari's age $+$ Gita's age $>$ Irfan's age $+$ Saira's age.
        \item The age difference between Gita and Saira is 1 year. However, Gita is not the oldest and Saira is not the youngest.
        \item There are no twins.
    \end{enumerate}
    In what order were they born (oldest first)?
    \begin{enumerate}
        \item HSIG
        \item SGHI
        \item IGSH
        \item HSGI
    \end{enumerate}
\end{enumerate}


  
\end{document}


