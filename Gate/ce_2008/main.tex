\let\negmedspace\undefined
\let\negthickspace\undefined
\documentclass[journal]{IEEEtran}
\usepackage[a5paper, margin=10mm, onecolumn]{geometry}
%\usepackage{lmodern} % Ensure lmodern is loaded for pdflatex
\usepackage{tfrupee} % Include tfrupee package

\setlength{\headheight}{1cm} % Set the height of the header box
\setlength{\headsep}{0mm}     % Set the distance between the header box and the top of the text

\usepackage{gvv-book}
\usepackage{gvv}
\usepackage{cite}
\usepackage{amsmath,amssymb,amsfonts,amsthm}
\usepackage{algorithmic}
\usepackage{graphicx}
\usepackage{textcomp}
\usepackage{xcolor}
\usepackage{txfonts}
\usepackage{listings}
\usepackage{enumitem}
\usepackage{mathtools}
\usepackage{gensymb}
\usepackage{comment}
\usepackage[breaklinks=true]{hyperref}
\usepackage{tkz-euclide} 
\usepackage{listings}
% \usepackage{gvv}                                        
\def\inputGnumericTable{}                                 
\usepackage[latin1]{inputenc}                                
\usepackage{color}                                            
\usepackage{array}                                            
\usepackage{longtable}                                       
\usepackage{calc}                                             
\usepackage{multirow}                                         
\usepackage{hhline}                                           
\usepackage{ifthen}                                           
\usepackage{lscape}
\begin{document}

\bibliographystyle{IEEEtran}
\vspace{3cm}

\title{CE - 2008}
\author{AI24BTECH11015 - Harshvardhan Patidar}
 \maketitle
% \newpage
% \bigskip
{\let\newpage\relax\maketitle}

\renewcommand{\thefigure}{\theenumi}
\renewcommand{\thetable}{\theenumi}
\setlength{\intextsep}{10pt} % Space between text and floats


\numberwithin{equation}{enumi}
\numberwithin{figure}{enumi}
\renewcommand{\thetable}{\theenumi}

\begin{enumerate}
    \item A road is provided with a horizontal circular curve having deflection angle of  $55^{\circ}$ and centre line radius of $250 m$. A transition curve is to be provided at each end of the circular curve of such a length that the rate of gain of radial acceleration is $0.3 m/s^3$ at a speed of $50 km$ per hour. Length of the transition curve required at each of the ends is
        \begin{enumerate}
            \item $2.57m$
            \item $33.33m$
            \item $35.73m$
            \item $1666.67m$
        \end{enumerate}

    \item A light house of $120 m$ height is just visible above the horizon from a ship. The corect distance $\brak{m}$ between the ship and the light house considering combined correction for curvature and refraction, is
        \begin{enumerate}
            \item $39.098$
            \item $42.226$
            \item $39098$
            \item $42226$
        \end{enumerate}

        \section*{Common Data Questions}
            \subsection*{Common Data for Questions \ref{71}, \ref{72} and \ref{73}:}
             
            A rectangular channel $6.0 m$ wide carries a discharge of $16.0 m^3 / s$ under uniform flow condition with normal depth fo $1.60 m$. Manning's $n$ is $0.015$.

            \item \label{71} The longitudinal slope of the channel is
                \begin{enumerate}
                    \item $0.000585$
                    \item $0.000485$
                    \item $0.000385$
                    \item $0.000285$
                \end{enumerate}
            \item \label{72}  A hump is to be provided on the channel bed. The maximum height of the hump without afecting the upstream flow condition is
                \begin{enumerate}
                    \item $0.50m$
                    \item $0.40m$
                    \item $0.30m$
                    \item $0.20m$
                \end{enumerate}

            \item \label{73} The channel width is to be contracted. The minimum width to which the channel can be contracted without affecting the upstream flow condition is
                \begin{enumerate}
                    \item $3.0m$
                    \item $3.8m$
                    \item $4.1m$
                    \item $4.5m$
                \end{enumerate}

            \subsection*{Common data for Questions \ref{74} and \ref{75}:}
            A reinforced concrete beam of rectangular cross section of breadth $230 mm$ and effective depth $400 mm$ is subjected to a maximum factored shear force of $120 $kN. The grades of concrete, main steel and stirrup steel are M$20$, Fe$415$ and Fe$250$ respectively. For the area of main steel provided, the design shear strength $\tau_c$ as per IS:$456-200$ is $0.48 N/mm^2$. The beam is designed for collapse limit state.
            \item \label{74} The spacing $\brak{mm}$ of $2$-legged $8 mm$ stirrups to be provided is
                \begin{enumerate}
                    \item $40$
                    \item $115$
                    \item $250$
                    \item $400$
                \end{enumerate}
            \item \label{75} In addition, the beam is subjected to a torque whose factored value is $10.90$ kN$m$. The stirrups have to be provided to carry a shear (kN) equal to
                \begin{enumerate}
                    \item $50.42$
                    \item $130.56$
                    \item $151.67$
                    \item $200.23$
                \end{enumerate}

    \section*{Linked Answer Questions: Q.\ref{76} to Q.\ref{85} carry two marks each.}
        \subsection*{Statement for Linked Answer Questions \ref{76} and \ref{77}:}
        Beam $GHI$ is supported by three pontoons as shown in the figure \ref{76-77}. The horizontal cross-sectional area of each pontoon is $8m^2$, the flexural rigidity of the beam is $10000$ kN-$m^2$ and the unit weight of water is $10$ kN$/m^3$.
        \begin{figure}[H]
            \centering
            \begin{circuitikz}[scale=0.25]
\tikzstyle{every node}=[font=\LARGE]
\draw [ fill={rgb,255:red,209; green,209; blue,209} , line width=1pt ] (-8.75,27) rectangle (-2.5,15.75);
\draw [ fill={rgb,255:red,209; green,209; blue,209} , line width=1pt ] (22.5,27) rectangle (28.75,15.75);
\draw [short] (-6.25,27) -- (-5.75,28.25);
\draw [short] (-5,27) -- (-5.75,28.25);
\draw [short] (25,27) -- (25.5,28.25);
\draw [short] (26.25,27) -- (25.5,28.25);
\draw [ line width=1.6pt](-5.75,28.25) to[short, -o] (0,28.25) ;
\draw [ line width=1.6pt](25.5,28.25) to[short, -o] (20,28.25) ;
\draw [ line width=1.6pt](0,28.25) to[short] (20,28.25);
\draw [line width=1pt, short] (-6.25,27) -- (-5.75,28.25);
\draw [line width=1pt, short] (-5,27) -- (-5.75,28.25);
\draw [line width=1pt, short] (25,27) -- (25.5,28.25);
\draw [line width=1pt, short] (26.25,27) -- (25.5,28.25);
\draw [line width=1pt, short] (5,28.25) -- (3.75,27);
\draw [line width=1pt, short] (5,28.25) -- (6.25,27);
\draw [line width=1pt, short] (10,28.25) -- (8.75,27);
\draw [line width=1pt, short] (10,28.25) -- (11.25,27);
\draw [line width=1pt, short] (13.75,27) -- (15,28.25);
\draw [line width=1pt, short] (15,28.25) -- (16.25,27);
\draw [ color={rgb,255:red,255; green,255; blue,255} , fill={rgb,255:red,247; green,247; blue,247}, line width=1pt ] (-2.5,24.5) rectangle (22.5,15.75);
\draw [ color={rgb,255:red,255; green,255; blue,255} , line width=1pt ] (30,23.25) rectangle (32,23.25);
\draw [ color={rgb,255:red,255; green,255; blue,255} , line width=1pt ] (9.5,18.75) rectangle (11.5,18.75);
\draw [ fill={rgb,255:red,209; green,209; blue,209} , line width=1pt ] (3.75,27) rectangle (6.25,20.75);
\draw [ fill={rgb,255:red,209; green,209; blue,209} , line width=1pt ] (8.75,27) rectangle (11.25,20.75);
\draw [ fill={rgb,255:red,209; green,209; blue,209} , line width=1pt ] (13.75,27) rectangle (16.25,20.75);
\node [font=\normalsize] at (10,18.25) {Pontoons};
\draw [line width=1pt, ->, >=Stealth] (8.5,19) -- (6.5,20.5);
\draw [line width=1pt, ->, >=Stealth] (9.75,19) -- (9.75,20.5);
\draw [line width=1pt, ->, >=Stealth] (11,19) -- (13.25,20.5);
\node [font=\normalsize] at (4.75,29.75) {G};
\node [font=\normalsize] at (10.75,29.75) {H};
\node [font=\normalsize] at (15.75,29.75) {I};
\draw [line width=1.3pt, ->, >=Stealth] (10,33.25) -- (10,28.25);
\draw [line width=1.3pt, ->, >=Stealth] (10,34.25) -- (12,34.25)node[pos=0.5, fill=white, font = \normalsize]{P = $48$ kN};
\draw [ line width=1.3pt](3.75,14.5) to[short] (16.25,14.5);
\draw [ line width=1.3pt](10,15.25) to[short] (10,14);
\draw [ line width=1.3pt](3.75,15.25) to[short] (3.75,14);
\draw [ line width=1.3pt](16.25,15.25) to[short] (16.25,14);
\node [font=\normalsize] at (6.75,13.75) {$5m$};
\node [font=\normalsize] at (12.75,13.75) {$5m$};
\end{circuitikz}

            \caption{}
            \label{76-77}
        \end{figure}
        \item \label{76} When the middle pontoon is removed, the deflection at $H$ will be
                \begin{enumerate}
                    \item $0.2m$
                    \item $0.4m$
                    \item $0.6m$
                    \item $0.8m$
                \end{enumerate}

            \item \label{77} When the midle pontoon is brought back to its position as shown in the figure \ref{76-77}, the reaction at $H$ will be
                \begin{enumerate}
                    \item $8.6$ kN
                    \item $15.7$ kN
                    \item $19.2$ kN
                    \item $24.2$ kN
                \end{enumerate}

    \subsection*{Statement for Linked Answer Questions \ref{78} and \ref{79}:}
    The ground conditions at a site are shown in the figure \ref{78-79}
    \begin{figure}[H]
        \centering
        \begin{circuitikz}[scale = 0.25]
\tikzstyle{every node}=[font=\LARGE]
\draw (-6.25,15.75) to[short] (32.5,15.75);
\draw [ line width=1.5pt](-6.25,15.75) to[short] (23.75,15.75);
\draw [ line width=1.1pt](-6.25,13.25) to[short] (-3.75,15.75);
\draw [ line width=1.1pt](-3.75,15.75) to[short] (-1.25,13.25);
\draw [ line width=1.1pt](-5,13.25) to[short] (-2.5,15.75);
\draw [ line width=1.1pt](-2.5,15.75) to[short] (0,13.25);
\draw [ line width=1.1pt](-3.75,13.25) to[short] (-1.25,15.75);
\draw [ line width=1.1pt](-1.25,15.75) to[short] (1.25,13.25);
\draw [ line width=1.1pt](23.75,13.25) to[short] (21.25,15.75);
\draw [ line width=1.1pt](21.25,15.75) to[short] (18.75,13.25);
\draw [ line width=1.1pt](22.5,13.25) to[short] (20,15.75);
\draw [ line width=1.1pt](20,15.75) to[short] (17.5,13.25);
\draw [ line width=1.1pt](21.25,13.25) to[short] (18.75,15.75);
\draw [ line width=1.1pt](18.75,15.75) to[short] (16.25,13.25);

\draw [ line width=1.1pt](27.5,15.75) to[short] (26.25,17);
\draw [ line width=1.1pt](28.75,17) to[short] (26.25,17);
\draw [ line width=1.1pt](27.5,15.75) to[short] (28.75,17);
\draw [ line width=1.1pt](26.25,15.5) to[short] (28.75,15.5);
\draw [ line width=1.1pt](26.5,15.5) to[short] (27.5,15.5);
\draw [ line width=1.1pt](26.5,15.25) to[short] (28.5,15.25);
\draw [ line width=1.1pt](26.75,15) to[short] (28.25,15);
\draw [ line width=1.1pt](27,14.75) to[short] (28,14.75);
\draw [ line width=1.1pt](27.25,14.5) to[short] (27.75,14.5);
\node [font=\large] at (12.75,17) {GL};
\node [font=\normalsize] at (8.75,14.75) {Sand};
\draw [line width=1.1pt, <->, >=Stealth] (5,15.75) -- (5,4.5);
\node at (6.25,4.5) [circ] {};
\node [font=\normalsize] at (6.75,5) {P};
\node [font=\normalsize] at (24.25,11.50) {Water table is at ground level};
\node [font=\normalsize] at (24.25,10.00) {Water content $= 20\%$};
\node [font=\normalsize] at (24.25,8.50) {Specific gravity of solids $=2.7$};
\node [font=\normalsize] at (24.25,7.00) {Unit weight of water $=10$kN$/m^2$};
\end{circuitikz}


        \caption{}
        \label{78-79}
    \end{figure}

    \item \label{78} The saturated unit weight of the sand (kN$/m^3$)
        \begin{enumerate}
            \item $15$
            \item $18$
            \item $21$
            \item $24$
        \end{enumerate}

    \item \label{79} The total stress, pore water pressure and effective stress (kN$/m^2$) at the point $\vec{P}$ are, respectively
        \begin{enumerate}
            \item $75,50$ and $25$
            \item $90,50$ and $40$
            \item $105,50$ and $55$
            \item $120,50$ and $70$
        \end{enumerate}
    \subsection*{Statement for Linked Answer Questions \ref{80} and \ref{81}:}
    A column is supported on a footing as shown in the figure \ref{80-81}. The water is at a depth of $10m$ below the base of the footing.
    \begin{figure}[H]
        \centering
        \begin{circuitikz}[scale=0.25]
\tikzstyle{every node}=[font=\normalsize]
\draw [ line width=1.3pt ] (7.5,40.75) rectangle (10,33.25);
\draw [line width=1.3pt, short] (7.5,35.75) -- (-7.5,35.75);
\draw [line width=1.3pt, short] (10,35.75) -- (25,35.75);
\node [font=\normalsize] at (13,40) {Column};
\node [font=\normalsize] at (21,36.50) {GL};
\draw [ line width=1.3pt ] (1.25,33.25) rectangle (16.25,27);
\node [font=\normalsize] at (8,30.25) {Footing};
\draw [line width=1.3pt, <->, >=Stealth] (17.5,27) -- (17.5,35.75);
\draw [line width=1.3pt, <->, >=Stealth] (16.25,25.75) -- (1.25,25.75);
\draw [line width=1.3pt, short] (1.25,26.5) -- (1.25,25);
\draw [line width=1.3pt, short] (16.25,26.5) -- (16.25,24.75);
\draw [line width=1.3pt, short] (16.75,27) -- (18.25,27);
\node [font=\normalsize] at (19,31) {$1.0m$};
\node [font=\normalsize] at (8.75,24.5) {$1.5m \times 3.0m$};
\draw [line width=1.3pt, short] (5,34.5) -- (3.75,35.75);
\draw [line width=1.3pt, short] (4.25,34.5) -- (3,35.75);
\draw [line width=1.3pt, short] (3,35.75) -- (1.75,34.5);
\draw [line width=1.3pt, short] (3.75,35.75) -- (2.5,34.5);
\draw [line width=1.3pt, short] (3.75,34.5) -- (5,35.75);
\draw [line width=1.3pt, short] (3.75,34.5) -- (2.5,35.75);
\draw [line width=1.3pt, short] (3,34.5) -- (4.25,35.75);
\draw [line width=1.3pt, short] (3,34.5) -- (1.75,35.75);
\draw [line width=1.3pt, short] (-5,34.5) -- (-3.75,35.75);
\draw [line width=1.3pt, short] (-4.25,34.5) -- (-3,35.75);
\draw [line width=1.3pt, short] (-3.75,35.75) -- (-2.5,34.5);
\draw [line width=1.3pt, short] (-3,35.75) -- (-1.75,34.5);
\draw [line width=1.3pt, short] (-3.75,34.5) -- (-5,35.75);
\draw [line width=1.3pt, short] (-3.75,34.5) -- (-2.5,35.75);
\draw [line width=1.3pt, short] (-3,34.5) -- (-1.75,35.75);
\draw [line width=1.3pt, short] (-3,34.5) -- (-4.25,35.75);
\draw [line width=1.3pt, short] (22.5,34.5) -- (21.25,35.75);
\draw [line width=1.3pt, short] (21.25,35.75) -- (20,34.5);
\draw [line width=1.3pt, short] (21.75,34.5) -- (20.5,35.75);
\draw [line width=1.3pt, short] (20.5,35.75) -- (19.25,34.5);
\draw [line width=1.3pt, short] (21.25,34.5) -- (22.5,35.75);
\draw [line width=1.3pt, short] (21.25,34.5) -- (20,35.75);
\draw [line width=1.3pt, short] (20.5,34.5) -- (21.75,35.75);
\draw [line width=1.3pt, short] (20.5,34.5) -- (19.25,35.75);
\draw [line width=1.3pt, short] (12.5,34.5) -- (13.75,35.75);
\draw [line width=1.3pt, short] (13.75,35.75) -- (15,34.5);
\draw [line width=1.3pt, short] (13.25,34.5) -- (14.5,35.75);
\draw [line width=1.3pt, short] (14.5,35.75) -- (15.75,34.5);
\draw [line width=1.3pt, short] (13.75,34.5) -- (12.5,35.75);
\draw [line width=1.3pt, short] (13.75,34.5) -- (15,35.75);
\draw [line width=1.3pt, short] (14.5,34.5) -- (15.75,35.75);
\draw [line width=1.3pt, short] (14.5,34.5) -- (13,35.75);
\node [font=\normalsize] at (28,32.5) {Sand};
\node [font=\normalsize] at (28,31.0) {$\gamma = 18$kN$/m^3$};
\node [font=\normalsize] at (28,29.5) {$N_q = 24$};
\node [font=\normalsize] at (28,28) {$N_{\gamma} = 20$};
\end{circuitikz}

        \caption{}
        \label{80-81}
    \end{figure}
    \item \label{80} The net ultimate bearing capacity (kN$/m^2$) of the footing based on Terzaghi's bearing capacity equation is
        \begin{enumerate}
            \item $216$
            \item $432$
            \item $630$
            \item $846$
        \end{enumerate}

    \item \label{81} The safe load (kN) that the footing can carry with a factor of safety $3$ is 
        \begin{enumerate}
            \item $282$
            \item $648$
            \item $945$
            \item $1269$
        \end{enumerate}

    \subsection*{Statement for Linked Answer Questions \ref{82} and \ref{83}:}
    An automobile with projected area $26m^2$ is running on a road with a speed of $120$ km per hour.The mass density and the kinematic viscosity of air are $1.2 kg/m^3$ and $1.5\times 10^{-5} m^2/s$, respectively. The drag coefficient is $0.30$.
    \item \label{82} The drag force on the automobile is
        \begin{enumerate}
            \item $620N$
            \item $600N$
            \item $580N$
            \item $520N$
        \end{enumerate}

    \item \label{83} The metric horse power required to overcome the drag force is 
        \begin{enumerate}
            \item $33.23$
            \item $31.23$
            \item $23.23$
            \item $20.23$
        \end{enumerate}

    \subsection*{Statement for Linked Answer Questions \ref{84} and \ref{85}:}
    A horizontal circular curve with a centre line radius of $200 m$ is provided on a $2$-lane, $2$-way SH section. The width of the $2$-lane road is $7.0 m$. Design speed for this section is $80$ km per hour. The brake reaction times is $2.4 s$, and the coefficients of friction in longitudinal and lateral directions are $0.355$ and $0.15$, respectively.
    \item \label{84} The safe stopping sight distance on the section is 
        \begin{enumerate}
            \item $221m$
            \item $195m$
            \item $125m$
            \item $65m$
        \end{enumerate}

    \item \label{85} The set-back distance from the centre line of the inner lane is 
        \begin{enumerate}
            \item $7.93m$
            \item $8.10m$
            \item $9.60m$
            \item $9.77m$
        \end{enumerate}
\end{enumerate}


  
\end{document}


