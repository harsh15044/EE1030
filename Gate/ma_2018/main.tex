\let\negmedspace\undefined
\let\negthickspace\undefined
\documentclass[journal]{IEEEtran}
\usepackage[a5paper, margin=10mm, onecolumn]{geometry}
%\usepackage{lmodern} % Ensure lmodern is loaded for pdflatex
\usepackage{tfrupee} % Include tfrupee package

\setlength{\headheight}{1cm} % Set the height of the header box
\setlength{\headsep}{0mm}     % Set the distance between the header box and the top of the text

\usepackage{gvv-book}
\usepackage{gvv}
\usepackage{cite}
\usepackage{amsmath,amssymb,amsfonts,amsthm}
\usepackage{algorithmic}
\usepackage{graphicx}
\usepackage{textcomp}
\usepackage{xcolor}
\usepackage{txfonts}
\usepackage{listings}
\usepackage{enumitem}
\usepackage{mathtools}
\usepackage{gensymb}
\usepackage{comment}
\usepackage[breaklinks=true]{hyperref}
\usepackage{tkz-euclide} 
\usepackage{listings}
% \usepackage{gvv}                                        
\def\inputGnumericTable{}                                 
\usepackage[latin1]{inputenc}                                
\usepackage{color}                                            
\usepackage{array}                                            
\usepackage{longtable}                                       
\usepackage{calc}                                             
\usepackage{multirow}                                         
\usepackage{hhline}                                           
\usepackage{ifthen}                                           
\usepackage{lscape}
\begin{document}

\bibliographystyle{IEEEtran}
\vspace{3cm}

\title{MA - 2018}
\author{AI24BTECH11015 - Harshvardhan Patidar}
 \maketitle
% \newpage
% \bigskip
{\let\newpage\relax\maketitle}

\renewcommand{\thefigure}{\theenumi}
\renewcommand{\thetable}{\theenumi}
\setlength{\intextsep}{10pt} % Space between text and floats


\numberwithin{equation}{enumi}
\numberwithin{figure}{enumi}
\renewcommand{\thetable}{\theenumi}

\section*{Q.\ref{1} - Q.\ref{5} carry one mark each}

\begin{enumerate}
    \item \label{1} "The dress \_\_\_\_\_\_\_ her so well that they all immediately \_\_\_\_\_\_\_ her on her appearance."\\ \\ The words that best fill the blanks in the above sentence are
        \begin{enumerate}
            \item complemented, complemented
            \item complimented, complemented
            \item complimented, complimented
            \item complemented, complimented
        \end{enumerate} 

    \item "The judge's standing in the legal community, though shaken by false allegations of wrongdoing, remained \_\_\_\_\_\_\_." \\ \\ The word that best fills the blank in the above sentence is 
        \begin{enumerate}
            \item undiminished
            \item damaged
            \item illegal
            \item uncertain
        \end{enumerate}

    \item Find the missing group of letters in the following series:\\ BC, FGH, LMNO, \_\_\_\_\_
        \begin{enumerate}
            \item UVWXY
            \item TUVWX
            \item STUVW
            \item RSTUV
        \end{enumerate}

    \item The perimeters of a circle, a square and an equilateral triangle are equal. Which one of the following statements is true?
        \begin{enumerate}
            \item The circle has the largest area.
            \item The square has the largest area.
            \item The equilateral triangle has the largest area.
            \item All the three shapes have the same area.
        \end{enumerate}

    \item \label{5} The value of the expression $\frac{1}{1 + \log_u vw} + \frac{1}{1 + \log_v wu} + \frac{1}{1 + \log_w uv}$ is \_\_\_\_\_\_\_.
        \begin{enumerate}
            \item $-1$
            \item $0$
            \item $1$
            \item $3$
        \end{enumerate}

\section*{Q.\ref{6} - Q.\ref{10} carry two marks each.}
    \item \label{6} Forty students watched films A, B and C over a week. Each student watched either only one film or all three. Thirteen students watched film A, sixteen students watched film B and nineteen students watched film C. How many students watched all three films?
        \begin{enumerate}
            \item $0$
            \item $2$
            \item $4$
            \item $8$
        \end{enumerate}

    \item A wire would enclose an area of $1936 m^2$, if it is bent into a square. The wire is cut into two pieces. The longer piece is thrice as long as the shorter piece. The long and the short pieces are bent into a square and a circle, respectively. Which of the following choices is closest to the sum of the areas enclosed by the two pieces in square meters?
        \begin{enumerate}
            \item $1096$
            \item $1111$
            \item $1243$
            \item $2486$
        \end{enumerate}
    
    \item A contract is to be completed in $52$ days and $125$ identical robots were employed, each operational for $7$ hours a day. After $39$ days, five-seventh of the work was completed. How many additional robots would be required to complete the work on time, if each robot is now operational for 8 hours a day?
        \begin{enumerate}
            \item $50$
            \item $89$
            \item $146$
            \item $175$
        \end{enumerate}

    \item A house has a number which needs to be identified. The following three statements are given that can help in identifying the house number.
        \begin{itemize}
            \item[i.] If the house number is a multiple of $3$, then it is a number from $50$ to $59$.
            \item[ii.] If the house number is NOT a multiple of $4$, then it is a number from $60$ to $69$.
            \item[iii.] If the house number is NOT a multiple of $6$, then it is a number from $70$ to $79$.
        \end{itemize}
        What is the house number?
        \begin{enumerate}
            \item $54$
            \item $65$
            \item $66$
            \item $76$
        \end{enumerate}

    \item \label{10} An unbiased coin is tossed six times in a row and four different such trials are conducted. One trial implies six tosses of the coin. If H stands for head and T stands for tail, the following are the observations from the four trials: \\ $\brak{1}$ HTHTHT $\brak{2}$ TTHHHT $\brak{3}$ HTTHHT $\brak{4}$ HHHT\_\_. \\ \\ Which statement describing the last two coin tosses of the fourth trial has the highest probability of being correct?
        \begin{enumerate}
            \item Two T will occur.
            \item One H and one T will occur.
            \item Two H will occur.
            \item One H will be followed by one T.
        \end{enumerate}

    \section*{Q.\ref{2s1} - Q.25 carry one mark each}
    \item \label{2s1} The principal value of $\brak{-1} ^ {\brak{-2i / \pi}}$ is
        \begin{enumerate}
            \item $e^{2}$
            \item $e^{2i}$
            \item $e^{-2i}$
            \item $e^{-2}$
        \end{enumerate}

    \item Let $f : \mathbb{C} \rightarrow \mathbb{C}$ be an entire function with $f\brak{0} = 1, f\brak{1} = 2 $ and $f^{\prime} \brak{0} = 0$. If there exists $M > 0$ such that $\abs{f^{\prime \prime} \brak{z}} \leq M$ for all $z \in \mathbb{C}$, then $f\brak{2} = $
        \begin{enumerate}
            \item $2$
            \item $5$
            \item $2+5i$
            \item $5 + 2 i$
        \end{enumerate}

    \item In the Laurent series expansion of $f\brak{z} =  \frac{1}{z\brak{z-1}}$ valid for $\abs{z-1} > 1$, the coefficient of $\frac{1}{z-1}$ is
        \begin{enumerate}
            \item $-2$
            \item $-1$
            \item $0$
            \item $1$
        \end{enumerate}
\end{enumerate}


  
\end{document}


