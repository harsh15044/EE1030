\documentclass{beamer}
\mode<presentation>
\usepackage{amsmath}
\usepackage{amssymb}
%\usepackage{advdate}
\usepackage{adjustbox}
\usepackage{subcaption}
\usepackage{enumitem}
\usepackage{multicol}
\usepackage{mathtools}
\usepackage{listings}
\usepackage{url}
% \usepackage{gvv}
\def\UrlBreaks{\do\/\do-}
\usetheme{Boadilla}
\usecolortheme{lily}
\setbeamertemplate{footline}
{
  \leavevmode%
  \hbox{%
  \begin{beamercolorbox}[wd=\paperwidth,ht=2.25ex,dp=1ex,right]{author in head/foot}%
    \insertframenumber{} / \inserttotalframenumber\hspace*{2ex} 
  \end{beamercolorbox}}%
  \vskip0pt%
}
\setbeamertemplate{navigation symbols}{}

\providecommand{\nCr}[2]{\,^{#1}C_{#2}} % nCr 
\providecommand{\nPr}[2]{\,^{#1}P_{#2}} % nPr
\providecommand{\mbf}{\mathbf}
\providecommand{\pr}[1]{\ensuremath{\Pr\left(#1\right)}}
\providecommand{\qfunc}[1]{\ensuremath{Q\left(#1\right)}}
\providecommand{\sbrak}[1]{\ensuremath{{}\left[#1\right]}}
\providecommand{\lsbrak}[1]{\ensuremath{{}\left[#1\right.}}
\providecommand{\rsbrak}[1]{\ensuremath{{}\left.#1\right]}}
\providecommand{\brak}[1]{\ensuremath{\left(#1\right)}}
\providecommand{\lbrak}[1]{\ensuremath{\left(#1\right.}}
\providecommand{\rbrak}[1]{\ensuremath{\left.#1\right)}}
\providecommand{\cbrak}[1]{\ensuremath{\left\{#1\right\}}}
\providecommand{\lcbrak}[1]{\ensuremath{\left\{#1\right.}}
\providecommand{\rcbrak}[1]{\ensuremath{\left.#1\right\}}}
\theoremstyle{remark}
\newtheorem{rem}{Remark}
\newcommand{\sgn}{\mathop{\mathrm{sgn}}}
\providecommand{\abs}[1]{\left\vert#1\right\vert}
\providecommand{\res}[1]{\Res\displaylimits_{#1}} 
\providecommand{\norm}[1]{\lVert#1\rVert}
\providecommand{\mtx}[1]{\mathbf{#1}}
\providecommand{\mean}[1]{E\left[ #1 \right]}
\providecommand{\fourier}{\overset{\mathcal{F}}{ \rightleftharpoons}}
%\providecommand{\hilbert}{\overset{\mathcal{H}}{ \rightleftharpoons}}
\providecommand{\system}{\overset{\mathcal{H}}{ \longleftrightarrow}}
	%\newcommand{\solution}[2]{\textbf{Solution:}{#1}}
%\newcommand{\solution}{\noindent \textbf{Solution: }}
\providecommand{\dec}[2]{\ensuremath{\overset{#1}{\underset{#2}{\gtrless}}}}
\newcommand{\myvec}[1]{\ensuremath{\begin{pmatrix}#1\end{pmatrix}}}
\let\vec\mathbf

\lstset{
%language=C,
frame=single, 
breaklines=true,
columns=fullflexible
}

\numberwithin{equation}{section}

\title{Matgeo Q.9.2.26}
\author{Harshvardhan Patidar - AI24BTECH11015\\Dept. of Artificial Engg.}

\date{\today} 
\begin{document}

\begin{frame}
\titlepage
\end{frame}

\section*{Outline}
\begin{frame}
\tableofcontents
\end{frame}
\section{Problem}
\begin{frame}
\frametitle{Problem Statement}
Find the area of the region included between $y^2 = 9x$ and $y=x$.
\end{frame}


%\subsection{Literature}
\section{Solution}


\subsection{Variables Used}
\begin{frame}
\frametitle{Variables Used}
  \begin{table}[ht]
    \begin{adjustbox}{width=0.8\framewidth}
     \begin{tabular}[12pt]{ |c| c|}
    \hline
    \textbf{Variable} & \textbf{Description}\\ 
    \hline
	$\vec{P}$ & \myvec{2a&a} point\\
    \hline
	$\vec{Q}$ & \myvec{2&-5} point\\
    \hline
	$\vec{R}$ & \myvec{-3&6} point\\
	\hline
	$\vec{a}$ & $y$-coordinate of $\vec{P}$\\
   \hline
    \end{tabular}

    \end{adjustbox}
    \vspace{0.5cm}
    \caption{Variables}
    \label{table}
  \end{table}
\end{frame}

\subsection{General equation of a Conic in Matrix form}
\begin{frame}
\frametitle{General equation of a Conic in Matrix Form}
  The general equation of a conic with directrix $\vec{n} ^{\top} \vec{x} = c$, Focus $\vec{F}$ and eccentricity $e$ is given by
  \begin{align}
		g\brak{\vec{x}}=\vec{x}^{\top}\vec{V}\vec{x}+2\vec{u}^{\top}\vec{x}+f=0 \label{gen_eq}
  \end{align}
  where,
  \begin{align}
		\vec{V}=\norm{\vec{n}}^2\vec{I}-e^2\vec{n}\vec{n}^{\top} \label{v_eq}\\
		\vec{u}=ce^2\vec{n}-\norm{\vec{n}}^2\vec{F}\\
		f=\norm{\vec{n}}^2\norm{\vec{F}}^2-c^2e^2
	\end{align} 
\end{frame}

\subsection{Parameters for given parabola}
\begin{frame}
  \frametitle{Parameters for given parabola}
  For the parabola $y = 9 x^2$, and,
  \begin{align}
    \text{directrix is} \myvec{-1&0}\vec{x} &= \frac{9}{4}\\
    \text{Focus } \vec{F} &= \myvec{\frac{9}{4}\\0}\\
    \text{and, eccentricity } e &= 1.
  \end{align}
\end{frame}

\subsection{Matrix Parameters of given Parabola}
\begin{frame}
  \frametitle{Matrix Parameters of given Parabola}
  We can now find $\vec{V}, \vec{u}, \text{ and } f$ to represent given parabola in the matrix equation form given in \ref{gen_eq}.\\
  From \ref{v_eq}, we have
  \begin{align}
    \vec{V} = \vec{I}  - \myvec{-1\\0} \myvec{-1&0} &\implies \vec{V} = \myvec{0&0\\0&1}\\
    \vec{u} = \frac{9}{4} \myvec{-1\\0} - \myvec{\frac{9}{4}\\0} &\implies \vec{u} = \myvec{-\frac{9}{2}\\0} \\
    f = \brak{\frac{9}{4}}^2 - \brak{\frac{9}{4}}^2  &\implies f = 0
  \end{align}
\end{frame}



\subsection{Line Parameters}
\begin{frame}
  \frametitle{Line Parameters}
  The given line $y=x$ can be represented in matrix form 
  \begin{align}
    \vec{x} = \vec{h} + \kappa \vec{m}
  \end{align}
  where
  \begin{align}
    \vec{h} = \myvec{0\\0} \\
    \vec{m} = \myvec{1\\1}
  \end{align}
\end{frame}

\subsection{Points of Intersection}
\begin{frame}
  \frametitle{Points of Intersection}
  The points of intersection of line and conic are given by
  \begin{align}
    \vec{x} = \vec{h} + \kappa_i \vec{m} \label{intersection}
  \end{align}
  where
  \begin{align}
    \kappa_{i} = \frac{1}{\vec{m}^{\top} \vec{V} \vec{m}} \brak{-\vec{m}^\top \brak{\vec{V} \vec{h} + \vec{u}} \pm \sqrt{\sbrak{\vec{m}^\top \brak{\vec{V} \vec{h} + \vec{u}}}^2 - g\brak{\vec{h}} \brak{\vec{m}^\top \vec{V} \vec{m}}}} \label{kappai}
  \end{align}
  On putting values in \ref{kappai}, we get 
  \begin{align}
    \kappa _ i = 0 , 9 \label{val_kappa}
  \end{align}
  Using \ref{val_kappa} in \ref{intersection}, we get points of intersection as $\myvec{0\\0}$ and $\myvec{9\\9}$
\end{frame}

\subsection{Calculating Area}
\begin{frame}
  \frametitle{Calculating Area}
  The area between the line and the parabola is given by
  % \begin{align}
  %   \int \limits_0^9 3 \sqrt{x} \,dx - \int \limits_0^9x \,dx  \notag \\
  % \end{align}
  \begin{align}
    \int \limits_0^9 3 \sqrt{x} \,dx - \int \limits_0^9 x \,dx  &= \brak{2 (9)^{3/2} - 2 (0)^{3/2}} - \brak{\frac{(9)^2}{2} - \frac{(0)^2}{2}} \notag \\
    &= \brak{2 \cdot 27 - 0} - \brak{\frac{81}{2} - 0} \notag \\
    &= \frac{27}{2} \label{final}
  \end{align}
  So, the area between the given parabola $y^2 = 9x$ and the line $y=x$ is $\frac{27}{2}$.
\end{frame}


% \begin{comment}
% The code in 
% {\footnotesize
% \begin{lstlisting}
% https://github.com/gadepall/school/blob/master/training/chemistry/codes/chembal.py
% \end{lstlisting}
% }
% \end{comment}

\end{document}
